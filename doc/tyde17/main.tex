%% For double-blind review submission
\documentclass[sigplan,review,anonymous]{acmart}\settopmatter{printfolios=true}
%% For single-blind review submission
%\documentclass[acmlarge,review]{acmart}\settopmatter{printfolios=true}
%% For final camera-ready submission
%\documentclass[acmlarge]{acmart}\settopmatter{}

%% Note: Authors migrating a paper from PACMPL format to traditional
%% SIGPLAN proceedings format should change 'acmlarge' to
%% 'sigplan,10pt'.


%% Some recommended packages.
\usepackage{booktabs}   %% For formal tables:
                        %% http://ctan.org/pkg/booktabs
\usepackage{subcaption} %% For complex figures with subfigures/subcaptions
                        %% http://ctan.org/pkg/subcaption
\usepackage{agda}
\usepackage{todonotes}
\usepackage{catchfilebetweentags}

\newcommand{\shgrab}[1]{\medskip\ExecuteMetaData[parser.tex]{#1}}
\newcommand{\hgrab}[1]{\ExecuteMetaData[parser.tex]{#1}}

\newcommand{\sagrab}[2]{\medskip\ExecuteMetaData[#1.tex]{#2}}
\newcommand{\agrab}[2]{\ExecuteMetaData[#1.tex]{#2}}
\newcommand{\BOX}{\AR{□}}


\usepackage{agda}
\usepackage{todonotes}
\usepackage{catchfilebetweentags}


%%%%%%%%%% AGDA ALIASES

\newcommand{\APT}{\AgdaPrimitiveType}
\newcommand{\AK}{\AgdaKeyword}
\newcommand{\AM}{\AgdaModule}
\newcommand{\AS}{\AgdaSymbol}
\newcommand{\AStr}{\AgdaString}
\newcommand{\AN}{\AgdaNumber}
\newcommand{\AD}{\AgdaDatatype}
\newcommand{\AF}{\AgdaFunction}
\newcommand{\AR}{\AgdaRecord}
\newcommand{\ARF}{\AgdaField}
\newcommand{\AB}{\AgdaBound}
\newcommand{\AIC}{\AgdaInductiveConstructor}

\newcommand{\Set}{\mathbf{Set}}

\newcommand{\remark}[1]{\paragraph{Remark: #1}}

\newcommand{\parser}[1]{\textit{#1}}
\newcommand{\type}[1]{\texttt{#1}}

\DeclareUnicodeCharacter{8759}{::}
\DeclareUnicodeCharacter{954}{$\kappa$}

\newcommand{\shgrab}[1]{\medskip\ExecuteMetaData[parser.tex]{#1}}
\newcommand{\hgrab}[1]{\ExecuteMetaData[parser.tex]{#1}}

\newcommand{\sagrab}[2]{\medskip\ExecuteMetaData[#1.tex]{#2}}
\newcommand{\agrab}[2]{\ExecuteMetaData[#1.tex]{#2}}
\newcommand{\BOX}{\AR{□}}

\input{lhs2tex-prelude.tex}

\input{lhs2tex-prelude.tex}

\makeatletter\if@ACM@journal\makeatother
%% Journal information (used by PACMPL format)
%% Supplied to authors by publisher for camera-ready submission
\acmJournal{PACMPL}
\acmVolume{1}
\acmNumber{1}
\acmArticle{1}
\acmYear{2017}
\acmMonth{1}
\acmDOI{10.1145/nnnnnnn.nnnnnnn}
\startPage{1}
\else\makeatother
%% Conference information (used by SIGPLAN proceedings format)
%% Supplied to authors by publisher for camera-ready submission
\acmConference[PL'17]{ACM SIGPLAN Conference on Programming Languages}{January 01--03, 2017}{New York, NY, USA}
\acmYear{2017}
\acmISBN{978-x-xxxx-xxxx-x/YY/MM}
\acmDOI{10.1145/nnnnnnn.nnnnnnn}
\startPage{1}
\fi


%% Copyright information
%% Supplied to authors (based on authors' rights management selection;
%% see authors.acm.org) by publisher for camera-ready submission
\setcopyright{none}             %% For review submission
%\setcopyright{acmcopyright}
%\setcopyright{acmlicensed}
%\setcopyright{rightsretained}
%\copyrightyear{2017}           %% If different from \acmYear


%% Bibliography style
\bibliographystyle{ACM-Reference-Format}
%% Citation style
%% Note: author/year citations are required for papers published as an
%% issue of PACMPL.
\citestyle{acmauthoryear}   %% For author/year citations



\newcommand{\parser}[1]{\texttt{#1}}
\newcommand{\type}[1]{\texttt{#1}}

\begin{document}

%% Title information
\title{agdarsec -- Total Parser Combinators}         %% [Short Title] is optional;
                                        %% when present, will be used in
                                        %% header instead of Full Title.
%\titlenote{with title note}             %% \titlenote is optional;
                                        %% can be repeated if necessary;
                                        %% contents suppressed with 'anonymous'
%\subtitle{Subtitle}                     %% \subtitle is optional
%\subtitlenote{with subtitle note}       %% \subtitlenote is optional;
                                        %% can be repeated if necessary;
                                        %% contents suppressed with 'anonymous'


%% Author information
%% Contents and number of authors suppressed with 'anonymous'.
%% Each author should be introduced by \author, followed by
%% \authornote (optional), \orcid (optional), \affiliation, and
%% \email.
%% An author may have multiple affiliations and/or emails; repeat the
%% appropriate command.
%% Many elements are not rendered, but should be provided for metadata
%% extraction tools.

%% Author with single affiliation.
\author{Guillaume Allais}
%\authornote{with author1 note}          %% \authornote is optional;
                                        %% can be repeated if necessary
%\orcid{nnnn-nnnn-nnnn-nnnn}             %% \orcid is optional
\affiliation{
%  \position{Position1}
  \department{Digital Security}              %% \department is recommended
  \institution{Radboud University}            %% \institution is required
%  \streetaddress{Street1 Address1}
  \city{Nijmegen}
%  \state{State1}
%  \postcode{Post-Code1}
  \country{The Netherlands}
}
\email{gallais@cs.ru.nl}          %% \email is recommended

%% Paper note
%% The \thanks command may be used to create a "paper note" ---
%% similar to a title note or an author note, but not explicitly
%% associated with a particular element.  It will appear immediately
%% above the permission/copyright statement.
% \thanks{with paper note}                %% \thanks is optional
                                        %% can be repeated if necesary
                                        %% contents suppressed with 'anonymous'


%% Abstract
%% Note: \begin{abstract}...\end{abstract} environment must come
%% before \maketitle command
\begin{abstract}
\end{abstract}


%% 2012 ACM Computing Classification System (CSS) concepts
%% Generate at 'http://dl.acm.org/ccs/ccs.cfm'.
\begin{CCSXML}
\end{CCSXML}

% \ccsdesc[500]{Software and its engineering~General programming languages}
% \ccsdesc[300]{Social and professional topics~History of programming languages}
%% End of generated code


%% Keywords
%% comma separated list
\keywords{Parser Combinators, Agda, Total Programming}  %% \keywords is optional


%% \maketitle
%% Note: \maketitle command must come after title commands, author
%% commands, abstract environment, Computing Classification System
%% environment and commands, and keywords command.
\maketitle


\section{Introduction}

Parser combinators have made functional languages such as Haskell
shine: they are a prime example of the advantages Embedded Domain
Specific Languages provide the end user. She not only has access
to a library of powerful composable abstractions but she is also
able to rely on the existing tooling and libraries of the host
language, she gets feedback from the static analysis built in the
compiler (e.g. type and coverage checking) and can exploit from
the expressivity of the host language to write generic parsers
thanks to polymorphism and higher order functions.


\section{A Primer on Parser Combinators}

\paragraph{Parser Type}

\shgrab{parser}

It is naturally possible to run such a parser and try to extract
a value from a successful parse. Opinions may vary on whether a
run with leftover characters can be considered successful or not.
We decide against it. This is not crucial as both styles can be
mutually emulated by either providing an 'end of file' parser
guaranteeing that only runs with no leftovers are successful or
by extending a grammar with a dummy case consuming the rest of
the input string.

\shgrab{parse}

\paragraph{(Strongly-Typed) Combinators}

The most basic parser is the one that accepts any character.
It succeeds as long as the input string is non empty and
returns one result: the tail of the string together with the
character it just read.

\shgrab{anychar}

However what makes parsers interesting is that they recognize
structure. As such, they need to reject invalid inputs. The
parser only accepting decimal digit is a bare bones example.
It can be implemented in terms of \parser{guard}, a higher
order parser checking that the value returned by its argument
abides by a predicate.

\shgrab{guard}
\hgrab{digit}

These two definitions are only somewhat satisfactory: the
resut of the \parser{digit} parser is still stringly-typed.
Instead of using a predicate to decide whether to keep the
value, we can opt for a validation function which returns
a witness whenever the check succeeds. To define this
alternative \parser{guardM} we can once more rely on code
already part of the standard library.

In our concrete example of recognizing a digit, we want to
return an \type{Int} corresponding to it. Once more the
standard library has just the right function to use together
with \parser{guardM}.

\shgrab{guard2}
\hgrab{digit2}

\paragraph{Code Reuse: Recognizing Structures}

We have seen how we can already rely on the standard library
of the host language to seemlessly implement combinators.
We can leverage even more of the existing codebase by noticing
that the type constructor \type{Parser} is a \type{Functor},
an \type{Applicative}~\todo{cite}, a \type{Monad} and also an \type{Alternative}.

% insert alternative instance

% interesting use case? replicateM?

\paragraph{Expressivity: Higher Order Parsers and Fixpoints}

Our first example of a higher order parser was \parser{guardM}
which takes as arguments a validation function as well as another
parser and produces a parser for the type of witnesses returned
by the validation function.

\shgrab{some}
\hgrab{many}

\section{The Issue with Haskell's Parser Types}

The ability to parse recursive grammars by simply declaring
them in a recursive manner is however dangerous: unlike type
errors which are caught by the typechecker and partial covers
in pattern matchings which are detected by the coverage checker,
termination is not guaranteed here.

The problem already shows up in the definition of \parser{some}
which will only make progress if its argument actually uses up
part of the input string. Otherwise it may loop. However this
is not the typical hurdle programmers stumble upon: demanding
a non empty list of nothing at all is after all rather silly.
The issue manifests itself naturally whenever defining a left
recursive grammar which leads us to introducing the prototypical
such example: \type{Expr}, a minimal language of arithmetic
expressions.

\[
\texttt{Expr} ~::=~ \texttt{<Int>} ~|~ \texttt{<Expr>} ~\text{`+`}~ \texttt{<Expr>}
\]

The intuitive solution is to simply reproduce this definition by
introducing an inductive type for \type{Expr} and then defining
the parser as an alternative between a literal on one hand and a
sub-expression, the character '+', and another sub-expression on
the other.

\shgrab{Expr}
\hgrab{expr}

However this leads to an infinite loop. Indeed, the second
alternative performs a recursive call to \parser{expr} even
though it hasn't consumed any character from the input string.

The typical solution to this problem is to introduce two 'tiers'
of expressions: the \parser{base} ones which can only be whole
expressions if we consume an opening parenthesis first and the
\parser{expr'} ones which are left-associated chains of \parser{base}
expressions connected by '+'.

\shgrab{base}
\hgrab{expr2}

This approach can be generalised when defining more complex
languages by having even more tiers, one for each
\emph{fixity level}. An extended language of arithmetic
expressions would for instance distinguish the level at
which addition and substraction live from the one at which
multiplication and division do.


Our issue with this solution is twofold. Firt, although we did
eventually managed to build a parser that worked as expected,
the compiler was unable to warn us and guide us towards this
correct solution. Additionnally, the blatant partiality of
some of these definitions means that these combinators and
these types are wholly unsuitable in a total. We could, of
course use an escape hatch and implement our parsers in Haskell
but that would both be unsafe and mean we would not be able
to run them at typechecking time which we may want to do if
we embed checked examples in our software's documentation,
or use constant values (e.g. filepaths).

\section{Indexed Sets and Course-of-Values Recursion}

Our implementation of Total Parser Combinators is in Agda,
a total dependently typed programming language and it will
rely heavily on indexed sets. But the indices will not be
playing any interesting role apart from enforcing totality.
As a consequence, we introduce combinators to build indexed
sets without having to mention the index explicitly. This
ought to make the types more readable by focusing on the
important components and hiding away the artefacts of the
encoding.

The first kind of combinator corresponds to operations on
sets which are lifted to indexed sets by silently propagating
the index. It may be a bit surprising to see infix notations
(in Agda underscores correspond to positions in which arguments
are to be inserted) for functions taking three arguments but
they are only meant to be partially applied.

\sagrab{Relation/Unary/Indexed}{arrow}
\agrab{Relation/Unary/Indexed}{product}

The second kind turns an indexed set into a set by universally
quantifying over the index. Here we use curly braces so that
the index we use is an \emph{implicit} argument we will never
have to write.

\sagrab{Relation/Unary/Indexed}{forall}

We can already see the benefits of using these aliases. For instance
the fairly compact expression \AF{[} (\AB{P} \AF{⊗} \AB{Q}) \AF{⟶} \AB{R} \AF{]}
corresponds to the more verbose type
\AS{∀} \{\AB{n}\} \AS{→} (\AB{P} \AB{n} \AF{×} \AB{Q} \AB{n}) \AS{→} \AB{R} \AB{n}.


Last but not least, the \BOX{} type constructor takes a \AD{ℕ}-indexed
set and produces the one corresponding to the allowed recursive
calls in a function defined by course-of-values recursion: \BOX{}
\AB{A} \AB{n} associates to each \AB{m} strictly smaller than \AB{n}
a value of type \AB{A} \AB{m}.

\sagrab{Induction/Nat/Strong}{box}

\paragraph{Remark: Record Wrapper} Instead of defining \BOX{} as a
function like the other combinators, we wrap the corresponding
function space in a record type. This prevents normalisation from
unfolding the combinator too eagerly and makes types more readable
during interactive development.

\paragraph{Remark: Irrelevance} The argument stating that \AB{m}
is strictly smaller than \AB{n} is preceded by a dot. In Agda,
it means that this value is irrelevant and can be erased by the
compiler. In Coq, we would define the relation \AF{\_<\_} in
\type{Prop} to achieve the same.

\medskip{}

This \BOX{} type combinator has some useful value combinators.
The first thing we can notice is the fact that \textbf{\BOX{} is a
functor}; that is to say that given a natural transformation from
\AB{A} to \AB{B}, we can define a natural transformation from
\BOX{} \AB{A} to \BOX{} \AB{B}.

\sagrab{Induction/Nat/Strong}{map}

\paragraph{Remark: Copatterns} The definition of \AF{map} uses
the \BOX{} field named \ARF{call} on the \emph{left hand side}.
This is a copattern\todo{cite}, meaning that we explain how
the definition is \emph{observed} (via \ARF{call}) rather than
\emph{constructed} (via \AIC{mkBox}).

\medskip{}

Because \AF{\_<\_} is defined in terms of \AD{\_≤\_}, \AF{≤-refl}
which is the proof that \AD{\_≤\_} is reflexive is also a proof
that any \AB{n} is strictly smaller than \AN{1} \AF{+} \AB{n}.
We can use this fact to write the following \AF{extract} function:

\sagrab{Induction/Nat/Strong}{extract}

\paragraph{Remark: Counit} The careful reader will have noticed
that this is not quite the extract we would expect from a comonad:
for a counit, we would need a natural transformation between
\BOX{} \AB{A} and \AB{A} i.e. a function of type
\AF{[} \BOX{} \AB{A} \AF{⟶} \AB{A} \AF{]}. We will not be able
to define such a function: \BOX{} \AB{A} \AN{0} is isomorphic to
the unit type so we would have to generate an \AB{A} \AN{0} out
of thin air.

\medskip

Even though we cannot have a counit, we are still able to define
a comultiplication thanks to the fact that \AF{\_<\_} is transitive.

\sagrab{Induction/Nat/Strong}{duplicate}

\agrab{Induction/Nat/Strong}{app}
\sagrab{Induction/Nat/Strong}{fix}

\paragraph{Remark: Generalisation} The type of \AF{fix} corresponds
exactly to strong induction for the natural numbers. A similar \BOX{}
type constructor can be defined for any induction principle relying
on an accessibility predicate. Which means that a library's types
can be cleaned up by using these combinators in any situation where
one had to give up structural induction for a more powerful alternative.

\section{Parsing, Totally}

%% Acknowledgments
%\begin{acks}                            %% acks environment is optional
                                        %% contents suppressed with 'anonymous'
  %% Commands \grantsponsor{<sponsorID>}{<name>}{<url>} and
  %% \grantnum[<url>]{<sponsorID>}{<number>} should be used to
  %% acknowledge financial support and will be used by metadata
  %% extraction tools.
%  This material is based upon work supported by the
%  \grantsponsor{GS100000001}{National Science
%    Foundation}{http://dx.doi.org/10.13039/100000001} under Grant
%  No.~\grantnum{GS100000001}{nnnnnnn} and Grant
%  No.~\grantnum{GS100000001}{mmmmmmm}.  Any opinions, findings, and
%  conclusions or recommendations expressed in this material are those
%  of the author and do not necessarily reflect the views of the
%  National Science Foundation.
%\end{acks}


%% Bibliography
%\bibliography{bibfile}


%% Appendix
%\appendix
%\section{Appendix}


\end{document}
